\documentclass[12pt,a4paper]{article}  %Вказівка класу документу
\usepackage[left=15mm,right=10mm,top=15mm,bottom=15mm]{geometry}   %Пакет задання нестандартних розмірів паперу
%\usepackage[cp1251]{inputenc}  %% 1
\usepackage[T2A]{fontenc}      %% Кодування шрифту української та російської мов
\usepackage[english, russian, ukrainian]{babel}    %% Коректна розстановка російських українських переносів (остання мова важається основною)
\usepackage[utf8x]{inputenc}
\usepackage{ucs}
\usepackage{amsmath}
\usepackage{amsthm}
\usepackage{amsfonts}
\usepackage{tikz}
\usepackage{amssymb}
\usepackage{indentfirst} %Відступи першого рядка абзацу
\usepackage{array}     %Макроси для малювання складних таблиць
\usepackage{longtable} %Багатосторінкові таблиці з повторенням шапки
\usepackage{setspace}  %Пакет міжстрокового інтервалу в пунктах
\usepackage{xcolor}    %Пакет виділення кольором тексту і фону
\usepackage{multicol}  %Пакет написання тексту в колонки
\usepackage{enumitem}  %Пакет для зручного переривання списку текстом
\usepackage{cite}      %Пакет коректного опряцювання списку літератури
\usepackage{xcolor}    %Колір тексту
\usepackage{caption}   %Зміна оформлення підписів плаваючих об'єктів
\usepackage{graphics}  %Пакет вставки зображень
\usepackage{multirow}  %Пакет об'єднання клітинок таблиці по вертикалі
\usepackage{comment}
\usepackage{makeidx}%алфавітний покажчик
\usepackage{lscape} % поддержка страниц в альбомной ориентации

\captionsetup[table]{labelsep=period, justification=raggedleft,singlelinecheck=false} %Зміна підпису табли згідно стандартам 
\setcounter{tocdepth}{3}\setcounter{secnumdepth}{3}
\author{Біляй Юрій Петрович}
\title{Контрольна робота з Теорії ймовірності та математичної статистики}
\setstretch{1.25}         %Міжрядковий інтервал у пунктах
\pagenumbering{arabic}  %Задання нумерації сторінок арабською нумерацією ()
% arabic — арабські цифри; 
% roman — малі римські цифри (i, ii, vi, ix); 
% Roman — Римські цифри (1, II, VI, IX); 
% alph — малі латинські букви (a, b, с); 
% Alph — Латинські букви (А, В, С); 
% asbuk — малі кириличні букви; 
% Asbuk — Кириличні букви. 
%\graphicspath{{images/eps/}{images/pdf/}}
% ++++++++++++++++++++++++++++++++++++

\newcolumntype{M}[1]{>{\centering\arraybackslash}m{#1}}
\newcolumntype{N}{@{}m{0pt}@{}}
\newcommand\wid{2.4cm}

%\pagestyle{empty}

\begin{document}
\section{Логарифмічні вирази}

\vspace{20pt}
\par\medskip \textbf{2007/о/1}\par
Розташуйте у порядку спадання числа $\sqrt{5}$; $2^{\log_2 5}$; $\dfrac{5}{2}$.

\begin{center}
\begin{tabular}{ |M{\wid}|M{\wid}|M{\wid}|M{\wid}|M{\wid}|N } 
 \hline
 \textbf{А} & \textbf{Б} & \textbf{В} & \textbf{Г} & \textbf{Д} & \\  [0.5em]
 \hline
 $2^{\log_2 5}$; $\dfrac{5}{2}$; $\sqrt{5}$ & $\dfrac{5}{2}$; $\sqrt{5}$; $2^{\log_2 5}$ & $\dfrac{5}{2}$;  $2^{\log_2 5}$; $\sqrt{5}$ & $\sqrt{5}$; $\dfrac{5}{2}$; $2^{\log_2 5}$ & $2^{\log_2 5}$; $\sqrt{5}$; $\dfrac{5}{2}$ &  \\ [1em]
 \hline
\end{tabular}
%\end{table}
\end{center}

\noindent\rule[0.5ex]{\linewidth}{1pt}

$2^{\log_2 5} = 5$~-- основна логарифмічна тотожність.
$\dfrac{5}{2} = 2,5$

Щоб порівняти всі числа піднесемо їх всі до квадрату:

$\sqrt{5}^2 = 5 < 2,5^2=6,25 < 5^2=25$
\textbf{(Г)}.

\vspace{20pt}
\par\medskip \textbf{2007/о/11}\par
Обчисліть $\log_{\frac{1}{25}}{\sqrt{5}}$

\begin{center}
\begin{tabular}{ |M{\wid}|M{\wid}|M{\wid}|M{\wid}|M{\wid}|N } 
 \hline
 \textbf{А} & \textbf{Б} & \textbf{В} & \textbf{Г} & \textbf{Д} & \\  [0.5em]
 \hline
 $-\dfrac{1}{4}$ & $-\dfrac{1}{2}$ & $-2$ & $\dfrac{1}{2}$ & $-\dfrac{1}{4}$ &  \\ [1em]
 \hline
\end{tabular}
%\end{table}
\end{center}

\noindent\rule[0.5ex]{\linewidth}{1pt}
$\log_{\frac{1}{25}}{\sqrt{5}} = \log_{5^{-2}}{5^{\frac{1}{2}}}=\dfrac{\frac{1}{2}}{-2}\log_5 5=-\dfrac{1}{4}$ \textbf{(Д)}.

\vspace{20pt}
\par\medskip \textbf{2007/о/12}\par
Розв'яжіть нерівність $\log_{0,1} 10 < \log_{0,1} x$.

\begin{center}
\begin{tabular}{ |M{\wid}|M{\wid}|M{\wid}|M{\wid}|M{\wid}|N } 
 \hline
 \textbf{А} & \textbf{Б} & \textbf{В} & \textbf{Г} & \textbf{Д} & \\  [0.5em]
 \hline
 $(10; +\infty)$ & $(0;10)$ & $(0,1;10)$ & $(-10;0)$ & $(-\infty; 10)$ &  \\ [1em]
 \hline
\end{tabular}
%\end{table}
\end{center}

\noindent\rule[0.5ex]{\linewidth}{1pt}
Спочатку визначимо область допустимих значень:
Підлогарифмічний вираз має бути більший за 0. $x > 0$.

Основа логарифма менше 1, тому під час порівняння підлогарифмічних виразів знак нерівності змінюється на протилежний:
$10 > x$, $x < 10$.

Врахувавши ОДЗ, остаточно отримаємо результат $0 < x < 10$, $x \in (0; 10)$ \textbf{(Б)}.

\vspace{20pt}
\par\medskip \textbf{2007/о/27}\par
Розв'яжіть систему рівнянь  
\begin{equation*}
\begin{cases}
   2^{2y-x}=32\\
   \log_{\frac{1}{2}}(y-x)=-2
\end{cases}
\end{equation*}

Запишіть у відповідь добуток $x_0\cdot y_0$, якщо пара $(x_0;y_0)$ є розв'язком вказаної системи рівнять.

\noindent\rule[0.5ex]{\linewidth}{1pt}

Область допустимих значень: $y-x>0 \Rightarrow x<y$.

\begin{equation*}
\begin{cases}
   2^{2y-x}=2^5\\
   \log_{\frac{1}{2}}(y-x)=\log_{\frac{1}{2}}{4}
\end{cases}
\end{equation*}

\begin{equation*}
\begin{cases}
   2y-x=5\\
   y-x=4
\end{cases}
\end{equation*}

\begin{equation*}
\begin{cases}
	y=1\\
	x=-3
\end{cases}
\end{equation*}

У відповідь запишемо \textbf{-3}.

\vspace{20pt}
\par\medskip \textbf{2007/о/29}\par
Обчисліть $\log_3 4 \cdot \log_{4} 5 \cdot \log_{5} 7 \cdot \log_7 81$.

\noindent\rule[0.5ex]{\linewidth}{1pt}

Використаємо властивість $\log_a b = \dfrac{1}{\log_{b} a}$, та $\dfrac{\log_c a}{\log_c b}=\log_b a$

$\log_3 4 \cdot \log_{4} 5 \cdot \log_{5} 7 \cdot \log_7 81$ = $\dfrac{\log_4 5}{\log_4 3} \cdot \log_{5} 7 \cdot \log_7 81$ = $\log_{3} 5 \cdot \log_{5} 7 \cdot \log_7 81$ = $\dfrac{\log_5 7}{\log_5 3} \cdot \log_7 81$ = $=\log_3 7 \cdot \log_7 81$ = $\dfrac{\log_7 81}{\log_7 3}$ = $\log_3 81 = 4$.

Відповідь: \textbf{4}.

\vspace{20pt}
\par\medskip \textbf{2007/о/32}\par

Знайдіть найменше ціле значення параметра $a$, при якому рівняння $\log_8{(x+2)}=\log_8{(2x-a)}$ має корені.

\noindent\rule[0.5ex]{\linewidth}{1pt}

ОДЗ: $2x-a>0 \Rightarrow x > \dfrac{a}{2}$

$x+2=2x-a$

$x = 2+a$

Врахуємо ОДЗ: $2+a>\dfrac{a}{2}$

$4+2a>a$

$a > -4$

Найменше ціле число $a$, що задовольняє обмеження~-- -3.

Відповідь: \textbf{-3}.

\newpage
\section{Теорія ймовірностей}

\vspace{20pt}
\par\medskip \textbf{2007/о/3}\par
 З натуральних чисел від 1 до 30 учень навмання називає одне. Яка ймовірність того, що це число є дільником 30?

\begin{center}
\begin{tabular}{ |M{\wid}|M{\wid}|M{\wid}|M{\wid}|M{\wid}|N } 
 \hline
 \textbf{А} & \textbf{Б} & \textbf{В} & \textbf{Г} & \textbf{Д} & \\  [0.5em]
 \hline
 $\dfrac{1}{30}$ & $\dfrac{2}{30}$ & $\dfrac{4}{15}$ & $\dfrac{6}{15}$ & $\dfrac{7}{15}$ &  \\ [1em]
 \hline
\end{tabular}
%\end{table}
\end{center}

\noindent\rule[0.5ex]{\linewidth}{1pt}

Кількість чисел, що є дільниками 30 від 1 до 30 є: 1,2,3,5,6,10,15,30. 
Їх кількість 8. Тому ймовірність дорівнює $\dfrac{8}{30}=\dfrac{4}{15}$ \textbf{(В)}

\vspace{20pt}
\par\medskip \textbf{2008/о/30}\par
У коробці є 80 цукерок, з яких 44~-- з чорного шоколаду, а решта~-- з білого. Визначте ймовірність того, що навмання взята цукерка з коробки буде з білого шоколаду.

\noindent\rule[0.5ex]{\linewidth}{1pt}

З білого шоколаду цукерок 80-44=32.
Ймовірність складатиме $\dfrac{32}{80}=\dfrac{2}{5}=\textbf{0,4}$.

\vspace{20pt}
\par\medskip \textbf{2009/о/13}\par

У туриста є 10 однакових за розмірами консервних банок, серед яких 4 банки -- з тушкованим м'ясом, 6 банок -- з рибою. Під час зливи етикетки відклеїлися. Турист навмання взяв одну банку. Яка ймовірність того, що вона буде з рибою?

\begin{center}
\begin{tabular}{ |M{\wid}|M{\wid}|M{\wid}|M{\wid}|M{\wid}|N } 
 \hline
 \textbf{А} & \textbf{Б} & \textbf{В} & \textbf{Г} & \textbf{Д} & \\  [0.5em]
 \hline
 $\dfrac{1}{10}$ & $\dfrac{2}{3}$ & $\dfrac{1}{6}$ & $\dfrac{3}{5}$ & $\dfrac{2}{5}$ &  \\ [1em]
 \hline
\end{tabular}
\end{center}

\noindent\rule[0.5ex]{\linewidth}{1pt}
Кількість банок з рибою -- 6. Всього банок -- 10. 
Ймовірність складатиме $\dfrac{6}{10}=\dfrac{3}{5}$ \textbf{(Г)}.

\vspace{20pt}
\par\medskip \textbf{2010/о1/15}\par

Пасічник зберігає мед в однакових закритих мталевих бідонах. Їх у нього дванадцять: у трьох бідонах міститься квітковий мед, у чотирьох~--мед з липи, у п'яти~--мед із гречки. Знайдіть імовірність того, що перший навмання відкритий бідон буде містити квітковий мед.

\begin{center}
\begin{tabular}{ |M{\wid}|M{\wid}|M{\wid}|M{\wid}|M{\wid}|N } 
 \hline
 \textbf{А} & \textbf{Б} & \textbf{В} & \textbf{Г} & \textbf{Д} & \\  [0.5em]
 \hline
 $\dfrac{1}{4}$ & $\dfrac{5}{12}$ & $\dfrac{1}{12}$ & $\dfrac{3}{4}$ & $\dfrac{1}{3}$ &  \\ [1em]
 \hline
\end{tabular}
\end{center}

\noindent\rule[0.5ex]{\linewidth}{1pt}
Кількість бідонів з квітковим медом -- 3. Всього бідонів -- 12. 
Ймовірність складатиме $\dfrac{3}{12}=\dfrac{1}{4}$ \textbf{(А)}.

\vspace{20pt}
\par\medskip \textbf{2010/о2/18}\par

На полиці знаходяться 18 однакових скляних банок із джемом. Серед них 6 банок з абрикосовим джемом, 12~--з яблучним. За кольором джеми не відрізняються один від одного. Господиня навмання взяла одну банку. Яка ймовірність того, що вона буде з абрикосовим джемом?

\begin{center}
\begin{tabular}{ |M{\wid}|M{\wid}|M{\wid}|M{\wid}|M{\wid}|N } 
 \hline
 \textbf{А} & \textbf{Б} & \textbf{В} & \textbf{Г} & \textbf{Д} & \\  [0.5em]
 \hline
 $\dfrac{1}{3}$ & $\dfrac{1}{6}$ & $\dfrac{2}{3}$ & $\dfrac{1}{18}$ & $\dfrac{1}{2}$ &  \\ [1em]
 \hline
\end{tabular}
\end{center}

\noindent\rule[0.5ex]{\linewidth}{1pt}
Кількість банок з абрикосовим варенням -- 6. Всього банок~-- 18. 
Ймовірність складатиме $\dfrac{6}{18}=\dfrac{1}{3}$ \textbf{(А)}.

\vspace{20pt}
\par\medskip \textbf{2010/п1/18}\par

На полиці розміщено 16 книг, з яких 6 книг~--історичні романи, а решта~--детективи. Знайдіть імовірність того, що перша книга, навмання взята з полиці, буде детективом.

\begin{center}
\begin{tabular}{ |M{\wid}|M{\wid}|M{\wid}|M{\wid}|M{\wid}|N } 
 \hline
 \textbf{А} & \textbf{Б} & \textbf{В} & \textbf{Г} & \textbf{Д} & \\  [0.5em]
 \hline
 $\dfrac{5}{8}$ & $\dfrac{1}{16}$ & $\dfrac{3}{5}$ & $\dfrac{1}{10}$ & $\dfrac{3}{8}$ &  \\ [1em]
 \hline
\end{tabular}
\end{center}

\noindent\rule[0.5ex]{\linewidth}{1pt}
Кількість детективів -- $16-6=10$. Всього книг -- 16. 
Ймовірність складатиме $\dfrac{10}{16}=\dfrac{5}{8}$ \textbf{(А)}.

\vspace{20pt}
\par\medskip \textbf{2010/п2/17}\par

У лотереї 10 виграшних білетів і 290 білетів без виграшу. Яка ймовірність того, що перший придбаний білет цієї лотереї буде виграшним?

\begin{center}
\begin{tabular}{ |M{\wid}|M{\wid}|M{\wid}|M{\wid}|M{\wid}|N } 
 \hline
 \textbf{А} & \textbf{Б} & \textbf{В} & \textbf{Г} & \textbf{Д} & \\  [0.5em]
 \hline
 $\dfrac{1}{29}$ & $\dfrac{29}{30}$ & $\dfrac{1}{300}$ & $\dfrac{1}{30}$ & $\dfrac{1}{10}$ &  \\ [1em]
 \hline
\end{tabular}
\end{center}

\noindent\rule[0.5ex]{\linewidth}{1pt}
Кількість виграшних білетів~-- 10. Всього білетів~-- $10+290=300$. 
Ймовірність складатиме $\dfrac{10}{300}=\dfrac{1}{30}$ \textbf{(Г)}.

\vspace{20pt}
\par\medskip \textbf{2011/о/31}\par

У відділі працює певна кількість чоловіків і жінок. Для анкетування навмання вибрали одного із співробітників. Імовірність того, що це чоловік, дорівнює $\dfrac{2}{7}$. Знайдіть відношення кількості жінок до кількості чоловіків, які працюють у цьому відділі.

\noindent\rule[0.5ex]{\linewidth}{1pt}

Якщо ймовірність обрати чоловіка дорівнює $\dfrac{2}{7}$, отже кількість чоловіків відноситься до всіх співробітників як 2 до 7, тому кількість жінок буде як 5 до 7. Відношення кількості жінок до кількості чоловіків дорівнюватиме $\dfrac{5}{2} = 2,5$.

\vspace{20pt}
\par\medskip \textbf{2013/о1/29}\par

В автобусному парку налічується $n$ автобусів, шосту частину яких було обладнано інформаціними табло. Пізніше інформаційні табло встановили ще на 4 автобуси з наявних у парку. Після проведеного переобладнання навмання вибирають один з $n$ автобусів парку. Ймовірність того, що це буде автобус з інформаційним табло, становить 0,25. Визначте $n$. Уважайте, що кожен автобус обладнується лише одним табло.

\noindent\rule[0.5ex]{\linewidth}{1pt}

Ймовірність 0,25 обрати автобус із інформаційним табло після переобладнання означає, що їх там $25\% = \dfrac{1}{4}$. Отже можна скласти рівняння $\dfrac{1}{6}n+4=\dfrac{1}{4}$, $2n+48=3n$, $n=\textbf{48}$.

\vspace{20pt}
\par\medskip \textbf{2013/о2/31}\par

У фестивалі беруть участь 25 гуртів, серед яких є по одному гурту з України і Чехії. Порядок виступу гуртів визначаєтся жеребкуванням, за яким кожен із гуртів має однакові шанси отримати будь-який порядковий номер від 1 до 25.
Знайдіть імовірність того, що на цьому фестивалі гурт з України виступатиме першим, а порядковий номер виступу гурту з Чехії буде парним.

\noindent\rule[0.5ex]{\linewidth}{1pt}
Ймовірність мати конкретний номер з 25 дорівнює $\dfrac{1}{25}$. Ймовірність мати парний номер дорівнює $\dfrac{12}{24}=\dfrac{1}{2}$. Оскільки ці події повинні виконуватись одночасно, то ймовірність шуканої події буде $\dfrac{1}{25}\cdot\dfrac{1}{2}=\dfrac{1}{50}=0,02=\textbf{2\%}$

\vspace{20pt}
\par\medskip \textbf{2013/п/29}\par

Студенти двох груп (у першій~-- 20 студентів, у другій~-- 25 студентів) обирають по одному представнику з кожної групи для участі в студнтському заході. Знайдіть ймовірність того, що учасниками заходу будуть обрані старости цих груп. Уважайте, що всі студенти кожної групи мають однакові шанси стати учасниками заходу, і в кожній групі є один староста.

\noindent\rule[0.5ex]{\linewidth}{1pt}
Ймовірність обрати страросту з кожної групи відповідно дорівнює $\dfrac{1}{20}$ та $\dfrac{1}{25}$. Оскільки ці події повинні виконуватись одночасно, то відповідь буде $\dfrac{1}{20}\cdot\dfrac{1}{25}=\dfrac{1}{500}=0,002=\textbf{0,2\%}$.

\vspace{20pt}
\par\medskip \textbf{2014/п/23}\par
З усіх натуральних чисел, більших за 9 і менших за 20, навмання вибирають одне число. Установіть відповідність між подією (1-4) та ймовірністю її появи (А-Д).

\begin{tabular}{ l l l l }
\multicolumn{2}{ l }{\textit{Подія} } & \multicolumn{2}{ l }{  \textit{Імовірність появи події} } \\ 
1 & вибране число буде простим & А & 0\\  
2 & вибране число буде двоцифровим & Б & 0,2\\  
3 & вибране число буде дільником числа 5 & В & 0,3\\  
4 & сума цифр вибраного числа буде ділитися на 3 & Г & 0,4\\  
 & & Д & 1\\  
\end{tabular}

\noindent\rule[0.5ex]{\linewidth}{1pt}
Кількість усіх чисел~-- 10. 

1) простих серед них (11, 13, 17, 19)~-- 4. Ймовірність буде становити $\dfrac{4}{10}=0,4$ \textbf{Г}

2) двоцифрових~-- 10. Ймовірність буде становити $\dfrac{10}{10}=1$ \textbf{Д}

3) дільників числа 5~-- 0. Ймовірність буде становити $\dfrac{0}{10}=0,2$ \textbf{А}

4) буде ділитись на 3~-- 3. Ймовірність буде становити $\dfrac{3}{10}=0,3$ \textbf{В} 

\vspace{20pt}
\par\medskip \textbf{2015/о/9}\par

Випущено партію з 300 лотерейних білетів. Імовірність того, що навмання вибраний білет із цієї партії буде виграшним, дорівнює 0,2. Визначте кількість білетів \textbf{без виграшу} серед цих 300 білетів.

\begin{center}
\begin{tabular}{ |M{\wid}|M{\wid}|M{\wid}|M{\wid}|M{\wid}|N } 
 \hline
 \textbf{А} & \textbf{Б} & \textbf{В} & \textbf{Г} & \textbf{Д} & \\  [0.5em]
 \hline
 6 & 60 & 294 & 150 & 240 & \\ [1em]
 \hline
\end{tabular}
\end{center}

\noindent\rule[0.5ex]{\linewidth}{1pt}
Кількість білетів з виграшом дорівнює $300\cdot0,2=60$. Кількість булетіів без виграшу буде складати $300-60=240$. \textbf{(Д)}.

\vspace{20pt}
\par\medskip \textbf{2015/д/9}\par

Кожну грань кубика пофабували або в синій, або в жовтий колір. Імовірність того, що при підкиданні кубика випаде синя грань, дорівнює $\dfrac{1}{3}$. Скільки всього граней кубика пофарбували в \textit{жовтий} колір?
\begin{center}
\begin{tabular}{ |M{\wid}|M{\wid}|M{\wid}|M{\wid}|M{\wid}|N } 
 \hline
 \textbf{А} & \textbf{Б} & \textbf{В} & \textbf{Г} & \textbf{Д} & \\  [0.5em]
 \hline
 п'ять & чотири & три & дві & одну & \\ [1em]
 \hline
\end{tabular}
\end{center}

\noindent\rule[0.5ex]{\linewidth}{1pt}
Всіх граней 6. Тому пофарбованих в фовтий колір буде $6\cdot(1-\dfrac{1}{3})=4$ \textbf{(Б)}.

\vspace{20pt}
\par\medskip \textbf{2015/п/11}\par
Майстер обслуговує лише три верстати: 20\% робочого часу він обслуговує перший верстат, 30\%~-- другий, 50\%~-- третій. Обчисліть імовірність того, що в навмання вибраний момент робочого часу майстер обслуговує перший або третій верстат.

\begin{center}
\begin{tabular}{ |M{\wid}|M{\wid}|M{\wid}|M{\wid}|M{\wid}|N } 
 \hline
 \textbf{А} & \textbf{Б} & \textbf{В} & \textbf{Г} & \textbf{Д} & \\  [0.5em]
 \hline
 0,8 & 0,7 & 0,5 & 0,3 & 0,1 & \\ [1em]
 \hline
\end{tabular}
\end{center}

\noindent\rule[0.5ex]{\linewidth}{1pt}
Перший або третій буде обслуговуватись з імовірністю $20\%+50\%=70\%=0,7$ \textbf{(Б)}.

\vspace{20pt}
\par\medskip \textbf{2016/п/8}\par
Комп'ютерна програма видаляє у восьмицифровому числі одну цифру навмання. Яка ймовірність того, що в числі 12506975 буде видалено цифру 5?

\begin{center}
\begin{tabular}{ |M{\wid}|M{\wid}|M{\wid}|M{\wid}|M{\wid}|N } 
 \hline
 \textbf{А} & \textbf{Б} & \textbf{В} & \textbf{Г} & \textbf{Д} & \\  [0.5em]
 \hline
 $\dfrac{5}{8}$ & $\dfrac{1}{8}$ & $\dfrac{1}{2}$ & $\dfrac{1}{5}$ & $\dfrac{1}{4}$ & \\ [1em]
 \hline
\end{tabular}
\end{center}

\noindent\rule[0.5ex]{\linewidth}{1pt}
Кількість всіх цифр~-- 8, а цифр "5"~-- 2.
Тому ймовірність можна обчислити $\dfrac{2}{8} = \dfrac{1}{4}$ \textbf{(Д)}.

\vspace{20pt}
\par\medskip \textbf{2017/о/29}\par
У торбинці лежать 3 цукерки з молочного шоколаду та $m$ цукерок з чорного шоколаду. Усі цукерки~-- однакової форми й розміру. Якого \textit{найменшого значення} може набувати $m$, якщо ймовірність навмання витягнути з торбинки цукерку з молочного шоколаду менша за $0,25$?

\noindent\rule[0.5ex]{\linewidth}{1pt}
Цукерок з молочним шоколадом~-- 3. Усіх цукерок~-- $m+3$. Ймовірність дістати цукерку з молочним шоколадом можна обчислити за формулою $\dfrac{3}{m+3}$.
Розв'яжемо відповідну нерівність $\dfrac{3}{m+3} < 0,25 = \dfrac{1}{4}=\dfrac{3}{12}$. Отже $m+3>12, m>9$. Оскільки нерівність строга, то найменше ціле число, що задовольняє дану нерівність буде \textbf{10}.

\vspace{20pt}
\par\medskip \textbf{2017/д/29}\par
Спортсмен робить один постріл у мішень. Імовірність того, що він влучить у мішень, у 7 разів більша за ймовірність того, що він у неї не влучить. Обчисліть імовірність того, що спортсмен влучить у мішень.

\noindent\rule[0.5ex]{\linewidth}{1pt}
Якщо розглядати лише дві можливі події: влучить і не влучить у мішень, то їх повна ймовірність буде 1. Нехай ймовірність того, що віне не влучить буде $x$, тоді ймовірність того, що він влучить буде~-- $7x$. Складемо відповідне рівнянн $x+7x=1$, $8x=1$, $x=\dfrac{1}{8}$. Ймовірність влучення буде $\dfrac{7}{8}=\textbf{0,875}$.

\newpage
\section{Комбінаторика}
\vspace{20pt}
\par\medskip \textbf{2008/о/13}\par
Укажіть, скільки можна скласти різних правильних дробів, чисельниками і знаменниками яких є числа 2, 3, 4, 5, 6, 7, 8, 9.

\begin{center}
\begin{tabular}{ |M{\wid}|M{\wid}|M{\wid}|M{\wid}|M{\wid}|N } 
 \hline
 \textbf{А} & \textbf{Б} & \textbf{В} & \textbf{Г} & \textbf{Д} & \\  [0.5em]
 \hline
 28 & 56 & 70 & 112 & Інша відповідь &  \\ [1em]
 \hline
\end{tabular}
\end{center}

\noindent\rule[0.5ex]{\linewidth}{1pt}

Для чисельника 2 можна підібрати всі інші 7 чисел щоб дріб був правильний, для 
3~-- лише 6, для 4~-- 5, ... . Знайдемо суму чисел 7+6+5+4+3+2+1=28 \textbf{(А)}.


\vspace{20pt}
\par\medskip \textbf{2009/о/18}\par
До складу української Прем'єр-ліги з футболу входять 16 команд. Упродовж сезону кожні дві команди грають між собою 2 матчі. Скільки всього матчів буде зіграно за сезон?

\begin{center}
\begin{tabular}{ |M{\wid}|M{\wid}|M{\wid}|M{\wid}|M{\wid}|N } 
 \hline
 \textbf{А} & \textbf{Б} & \textbf{В} & \textbf{Г} & \textbf{Д} & \\  [0.5em]
 \hline
 120 & 128 & 200 & 240 & 256 &  \\ [1em]
 \hline
\end{tabular}
\end{center}

\noindent\rule[0.5ex]{\linewidth}{1pt}

Кожна з 16 команд зіграє з іншими 15, кількість матчів буде $15\cdot 16 = 240$. Оскільки в такий спосіб будо пораховано кожен матч двічі (для кожної команди), тому загальна кількість матчів буде $\dfrac{240}{2}=120$ \textbf{(А)}

\vspace{20pt}
\par\medskip \textbf{2010/о1/23}\par

Студенти однієї з груп під час сесії повинні скласти п'ять іспитів. Заступнику декана потрібно призначити складання цих іспитів на п'ять визначених дат. Скільки всього варіантів розкладу іспитів для цієї групи?

\begin{center}
\begin{tabular}{ |M{\wid}|M{\wid}|M{\wid}|M{\wid}|M{\wid}|N } 
 \hline
 \textbf{А} & \textbf{Б} & \textbf{В} & \textbf{Г} & \textbf{Д} & \\  [0.5em]
 \hline
 5 & 25 & 60 & 120 & 240 &  \\ [1em]
 \hline
\end{tabular}
\end{center}

\noindent\rule[0.5ex]{\linewidth}{1pt}

Оскільки порядок іспитів є важливим і обираються всі 5 іспитів з 5, то застосовуємо формулу перестановок $P_n=n!$. $P_5=5!=5\cdot4\cdot3\cdot2\cdot1=120$ \textbf{(Г)}.

\vspace{20pt}
\par\medskip \textbf{2010/о2/21}\par

Кодовий замок на дверях має десять кнопок, на яких нанесено десять різних цифр (див. рисунок).

\begin{center}
\begin{tikzpicture}
\draw (0,0) rectangle (3.6,1.5);
\draw (0.4,1.1) circle (0.3);
\node at (0.4,1.1) {$1$};
\draw (0.4,0.4) circle (0.3);
\node at (0.4,0.4) {$6$};
\draw (1.1,1.1) circle (0.3);
\node at (1.1,1.1) {$2$};
\draw (1.1,0.4) circle (0.3);
\node at (1.1,0.4) {$7$};
\draw (1.8,1.1) circle (0.3);
\node at (1.8,1.1) {$3$};
\draw (1.8,0.4) circle (0.3);
\node at (1.8,0.4) {$8$};
\draw (2.5,1.1) circle (0.3);
\node at (2.5,1.1) {$4$};
\draw (2.5,0.4) circle (0.3);
\node at (2.5,0.4) {$9$};
\draw (3.2,1.1) circle (0.3);
\node at (3.2,1.1) {$5$};
\draw (3.2,0.4) circle (0.3);
\node at (3.2,0.4) {$0$};
\end{tikzpicture}
\end{center}
 
 Щоб відчинити двері, потрібно одночасно натиснути дві кнопки, цифри на яких складають код замка. Скільки всього існує різних варіантів коду замка? Уважайте, що коди, утворені перестановкою цифр (наприклад, 1-2 і 2-1), є однаковими.



\begin{center}
\begin{tabular}{ |M{\wid}|M{\wid}|M{\wid}|M{\wid}|M{\wid}|N } 
 \hline
 \textbf{А} & \textbf{Б} & \textbf{В} & \textbf{Г} & \textbf{Д} & \\  [0.5em]
 \hline
 100 & 90 & 45 & 20 & 10 &  \\ [1em]
 \hline
\end{tabular}
\end{center}

\noindent\rule[0.5ex]{\linewidth}{1pt}

Оскільки порядок кнопок не є важливим і обираються лише 2 кнопки з 10, то то застосовуємо формулу числа комбінацій $C_n^m=\dfrac{n!}{m!\cdot(n-m)!}$.

$C_{10}^2=\dfrac{10!}{2!\cdot(10-2)!}=\dfrac{10\cdot9}{2\cdot1}=45$ \textbf{(В)}.

\vspace{20pt}
\par\medskip \textbf{2010/п1/24}\par

У кіоску є 10 видів вітальних листівок з Новим роком. Скільки всього можна утворити різних наборів листівок, кожен із яких складається з трьох листівок різних видів?

\begin{center}
\begin{tabular}{ |M{\wid}|M{\wid}|M{\wid}|M{\wid}|M{\wid}|N } 
 \hline
 \textbf{А} & \textbf{Б} & \textbf{В} & \textbf{Г} & \textbf{Д} & \\  [0.5em]
 \hline
 30 & 90 & 120 & 240 & 720 &  \\ [1em]
 \hline
\end{tabular}
\end{center}

\noindent\rule[0.5ex]{\linewidth}{1pt}

Оскільки порядок листівок не є важливим, то застосовуємо формулу числа комбінацій $C_n^m=\dfrac{n!}{m!\cdot(n-m)!}$.

$C_{10}^3=\dfrac{10!}{3!\cdot(10-3)!}=\dfrac{10\cdot9\cdot8}{3\cdot2\cdot1}=120$ \textbf{(В)}.

\vspace{20pt}
\par\medskip \textbf{2010/п2/23}\par

Скільки всього різних п'ятицифрових чисел можна утворити з цифр 0, 1, 3, 5, 7 (у числах цифри не повинні повторюватися)?

\begin{center}
\begin{tabular}{ |M{\wid}|M{\wid}|M{\wid}|M{\wid}|M{\wid}|N } 
\hline
\textbf{А} & \textbf{Б} & \textbf{В} & \textbf{Г} & \textbf{Д} & \\  [0.5em]
\hline
5 & 24 & 25 & 96 & 120 &  \\ [1em]
\hline
\end{tabular}
\end{center}

\noindent\rule[0.5ex]{\linewidth}{1pt}
На першу позицію можна поставити одну з чотирьох цифр, на другу~-- 4, на третю~--3, .... Кількість чисел дорівнює $4\cdot4\cdot3\cdot2\cdot1=96$ \textbf{(Г)}.

\vspace{20pt}
\par\medskip \textbf{2011/п/31}\par

Заступник директора школи складає розклад уроків для 10-го класу. Він запланував на понеділок шість уроків з таких предметів: геометрія, біологія, англійська мова, хімія, фізична культура, географія. Скільки всього існує різних варіантів розкладу уроків на цей день, якщо урок фізичної культури має бути останнім у розкладі?

\noindent\rule[0.5ex]{\linewidth}{1pt}

Оскільки місце фізичної культури визначене, то його можна не враховувати. Залишається 5 предметів для яких важливий порядок, то застосовуємо формулу перестановок $P_n=n!$. $P_5=5!=5\cdot4\cdot3\cdot2\cdot1=\textbf{120}$.

\vspace{20pt}
\par\medskip \textbf{2012/п/28}\par

Скільки всього існує різних двоцифрових чисел, у яких перша цифра є парною, а друга~-- непарною?

\noindent\rule[0.5ex]{\linewidth}{1pt}

На місце десятих можна обрати одну з чотирьох парних чисел, а на місце одиниць~-- одну з п'яти. Тому кількість тахких чисел буде $4\cdot5=\textbf{20}$.

\vspace{20pt}
\par\medskip \textbf{2012/о1/26}\par

Скільки існує різних дробів $\frac{m}{n}$, якщо $m$ набуває значень 1; 2 або 4, а $n$ набуває значень 5; 7; 11; 13 або 17?

\noindent\rule[0.5ex]{\linewidth}{1pt}

Оскільки для будь яких комбінацій не отримуються скоротні дроби, то загальна кількість буде $3\cdot5=\textbf{15}$.

\vspace{20pt}
\par\medskip \textbf{2012/о2/26}\par

Скільки всього різних двоцифрових чисел можна утворити з цифр 1, 5, 7 і 8 так, щоб у кожному числі всі цифри не повторювались.

\noindent\rule[0.5ex]{\linewidth}{1pt}

Оскільки порядок цифр є важливим, то застосовуємо формулу числа розміщень $A_n^m=\dfrac{n!}{(n-m)!}$.

$A_{4}^2=\dfrac{4!}{(4-2)!}=\dfrac{4\cdot3\cdot2\cdot1}{2\cdot1}=\textbf{12}$.

\vspace{20pt}
\par\medskip \textbf{2014/о/6}\par
Студент на першому курсі повинен вибрати одну з трьох іноземних мов, яку вивчатиме, та одну з п'яти спортивних секцій, що відвідуватиме. Скільки всього існує варіантів вибору студентом іноземної мови та спортивної секції?

\begin{center}
\begin{tabular}{ |M{\wid}|M{\wid}|M{\wid}|M{\wid}|M{\wid}|N } 
 \hline
 \textbf{А} & \textbf{Б} & \textbf{В} & \textbf{Г} & \textbf{Д} & \\  [0.5em]
 \hline
 5 & 8 & 10 & 15 & 28 &  \\ [1em]
 \hline
\end{tabular}
\end{center}

\noindent\rule[0.5ex]{\linewidth}{1pt}
Оскільки вибір повиннен виконуватись одночасно, то всього варіантів буде $3\cdot5=15$ \textbf{(Г)}.

\vspace{20pt}
\par\medskip \textbf{2014/д/4}\par
Блок соціальної реклами складається з 4 рекламних роликів: про шкідливість паління, про охорону навколишнього середовища, про дотримання правил дорожнього руху та про велосипедне місто. Ролик про шкідливість паління заплановано показати двічі~-- першим і останнім, а інші три ролики~-- по одному разу. Скільки всього існує варіантів формування цього блоку соціальної реклами за вказаним порядком рекламних роликів?

\begin{center}
\begin{tabular}{ |M{\wid}|M{\wid}|M{\wid}|M{\wid}|M{\wid}|N } 
 \hline
 \textbf{А} & \textbf{Б} & \textbf{В} & \textbf{Г} & \textbf{Д} & \\  [0.5em]
 \hline
 6 & 8 & 12 & 24 & 120 &  \\ [1em]
 \hline
\end{tabular}
\end{center}

\noindent\rule[0.5ex]{\linewidth}{1pt}
У блоку реклами не звизначені 3 місця для трьох роликів, тому кількість варіантів можна обчислити за формулою числа перестановок $P_n=n!$. $P_3=3!=3\cdot2\cdot1=6$ \textbf{(A)}.


\vspace{20pt}
\par\medskip \textbf{2016/о/30}\par
У чайному кіоску в наявності є лише розфасований у коробки по 100г листовий чорний чай 8 видів, серед яких є вид "чорна перлина". Покупець вирішив придбати в цьому кіоску для подарункового набору при коробки чорного чоаю трьох річзних видів, серед яких обов'язково повинен бути вид "чорна перлина". Скільки всього в покупця є варіантів такого вридбання трьох коробок чаю для набору з наявних у кіоску?

\noindent\rule[0.5ex]{\linewidth}{1pt}
Оскільки вид "чорна перлина" уже зафіксовано, то залишається 7 видів на дві позиції, де порядок не важливий, тому використовуємо формулу числа комбінацій 

$C_n^m=\dfrac{n!}{m!\cdot(n-m)!}$.
$C_{7}^2=\dfrac{7!}{2!\cdot(7-2)!}=\dfrac{7\cdot6}{2\cdot1}=\textbf{21}$.

\vspace{20pt}
\par\medskip \textbf{2016/д/29}\par
У магазині в наявності є 10 видів тортів та 15 видів пачок печива. Скільки вього є способів вибору в цьому магазині \textit{або} одного торта, \textit{або} трьох різних пачок печива для всяткового вечора?

\noindent\rule[0.5ex]{\linewidth}{1pt}
Вибір торту виключає можливість вибору печива і навпаки, порядок вибору видів печива не важливий~-- використовуємо формулу числа комбінацій і формулу додавання: $C_n^m=\dfrac{n!}{m!\cdot(n-m)!}$. 
$C_{10}^1 + C_{15}^3 = \dfrac{10!}{1!\cdot(10-1)!} + \dfrac{15!}{3!\cdot(15-3)!}=\dfrac{10}{1} + \dfrac{15\cdot14\cdot13}{3\cdot2\cdot1}=10+455=\textbf{465}$.

\vspace{20pt}
\par\medskip \textbf{2016/п/29}\par
Марійка зірвала на клумбі 9 нарцисів на 4 тюльпани. Скільки всього існує способів вибору із цих квітів 3 нарцисів та 2 тюльпанів для букета?

\noindent\rule[0.5ex]{\linewidth}{1pt}
Кількість способів обрати 3 нарциси з 9 можна обчислити за формулою числа комбінацій (порядок не має значення).
$C_n^m=\dfrac{n!}{m!\cdot(n-m)!}$.
$C_{9}^3=\dfrac{9!}{3!\cdot(9-3)!}=\dfrac{9\cdot8\cdot7}{3\cdot2\cdot1}=84$.
Аналогічно обсилюється кількість варіантів вибору 2 тюльпанів з чотирьох можливих.
$C_{4}^2=\dfrac{4!}{2!\cdot(4-2)!}=\dfrac{4\cdot3}{2\cdot1}=6$.
Оскільки до букету потрібно додати і нарциси і тюльпани, то застосовуємо формулу множення $84\cdot6=\textbf{504}$.

\vspace{20pt}
\par\medskip \textbf{2017/п/29}\par
Музей має надати чотири картини відомого художника для виставки, прсвяченох дню його народження. Одну картину вибирають з діючої експозицї музею, що містить 5 робіт цього художника, а три інші~-- з архіву, у якому є 10 його картин. Скільки всього способів такого вибору?

\noindent\rule[0.5ex]{\linewidth}{1pt}
Кількість способів обрати 1 картину діючої експозиції з 5 можна обчислити за формулою числа комбінацій (порядок не має значення).
$C_n^m=\dfrac{n!}{m!\cdot(n-m)!}$.
$C_{5}^1=\dfrac{5!}{1!\cdot(5-1)!}=\dfrac{5}{1}=5$.
Аналогічно обсилюється кількість варіантів вибору 3 картин з 10 архівних.
$C_{10}^3=\dfrac{10!}{3!\cdot(10-3)!}=\dfrac{10\cdot9\cdot8}{3\cdot2\cdot1}=120$.
Оскільки подати потрібно одночасно і картини з діючої виставки і із архіву, то застосовуємо формулу множення $5\cdot120=\textbf{600}$.

\vspace{20pt}
\par\medskip \textbf{2018/о/29}\par
В Оленки є 8 різних фотографій з її зображенням та 6 різних фотографій її класу. Скільки всього в неї є способів вибрати з них 3 фотографії зі своїм зображенням для персональної сторінки в соціальній мережі та 2 фотографії свого класу для сайту школи?

\noindent\rule[0.5ex]{\linewidth}{1pt}
Кількість способів обрати 3 фотографії з 8 можна обчислити за формулою числа комбінацій (порядок не має значення).
$C_n^m=\dfrac{n!}{m!\cdot(n-m)!}$.
$C_{8}^3=\dfrac{8!}{3!\cdot(8-3)!}=\dfrac{8\cdot7\cdot6}{3\cdot2\cdot1}=56$.
Аналогічно обсилюється кількість варіантів вибору 2 фотографій класу з 6 можливих.
$C_{6}^2=\dfrac{6!}{2!\cdot(6-2)!}=\dfrac{6\cdot5}{2\cdot1}=15$.
Оскільки обрати потрібно одночасно і обидва типи фотограіфй, то застосовуємо формулу множення $56\cdot15=\textbf{840}$.

\vspace{20pt}
\par\medskip \textbf{2018/д/29}\par
Піцерія пропонує послугу "Зроби піцу сам", що передбачає вибір клієнтом добавок для піци? Поміж добавок~-- 8 м'ясних (шинка, ковбаса та інші) і 9 овочевих (цибуля, перець та інші). Клієнт вибирає 2 м'ясні добавки, однією з яких обов'язково має бути шинка, і 3~-- овочевих, за винятком цибулі. Скільки всього існує варіантів такого вибору клієнтом?

\noindent\rule[0.5ex]{\linewidth}{1pt}
Оскільки з м'ясних добавок одна зафіксована, то залишається обрати одну з 7, що залишились. Кількість таких варінтів вибору буде $C_{7}^1=\dfrac{7!}{1!\cdot(7-1)!}=\dfrac{7}{1}=7$.
Оскільки з овочевих добавок не можна обирати цибулю, то фактично для вибору залишається 8 овочевих. Аналогічно отримуємо $C_{8}^3=\dfrac{8!}{3!\cdot(8-3)!}=\dfrac{8\cdot7\cdot6}{3\cdot2\cdot1}=56$.
Вибір добавок відбувається одночасно, тому використовуємо формулу множення $7\cdot56=\textbf{392}$.

\vspace{20pt}
\par\medskip \textbf{2018/п/29}\par
Для перевезення дітей формують колону, яка складається з п'яти автобусів і двох супровідних автомобілів: одного на чолі колони, іншого~-- позаду неї. Скільки всього існує різних способів розташування автобусів і супровідних автомобілів у цій колоні?

\noindent\rule[0.5ex]{\linewidth}{1pt}
В даному випадку обираються всі елементи з усіх можливих і порядок має значення, тому використовуємо формулу числа перестановок $P_n=n!$.
Кількість способів розташувати автобуси у кололі буде $P_5=5!=5\cdot4\cdot3\cdot2\cdot1=120$. Кількість способів розташувати автомобілі у кололі буде $P_2=2!=2\cdot1=2$. Транспортні засоби одночасно обираються у колону, тому використовуємо формулу множення. $120\cdot2=\textbf{240}$.

\vspace{20pt}
\par\medskip \textbf{2019/о/29}\par
У фінал пісенного конкурсу вийшло 4 солісти та 3 гурти. Порядковий номер виступу фіналістів визначають жеребкуванням. Скільки вього є варіантів послідовностей виступів фіналістів, якщо спочатк виступатимуть гурти, а після них~-- солісти?
Уважайте, що кожен фіналіст виступатиме у фіналі лише один раз.

\noindent\rule[0.5ex]{\linewidth}{1pt}
В даному випадку обираються всі елементи з усіх можливих і порядок має значення, тому використовуємо формулу числа перестановок $P_n=n!$.
Кількість способів розташувати гурти для виступу буде $P_3=3!=3\cdot2\cdot1=6$. Кількість способів розташувати солістів для виступу буде $P_4=4!=4\cdot3\cdot2\cdot1=24$. Усі учасники братимуть участь у фіналі конкурсу, тому використовуємо формулу множення. $6\cdot24=\textbf{144}$.

\vspace{20pt}
\par\medskip \textbf{2019/д/29}\par
У магазині в продажу є 6 видів тарілок, 8 видів блюдець та 12 видів чашок. Олена збирається купити бабусі в подарунок у цьому магазині або чашку та блюдце, або лише тарілку. Скільки вього є способів в Оленку купити бабусі такий подарунок?

\noindent\rule[0.5ex]{\linewidth}{1pt}
Способів купити чашку та блюдце~-- $8\cdot12=96$. Способів купити тарілку~-- $6$. Оскільки для вибору подарунку підходить лише один з варіантів~-- використовуємо формулу додавання $96+6=\textbf{102}$.

\vspace{20pt}
\par\medskip \textbf{2019/п/29}\par
Для оформлення салону краси врішили замовити в магазині квітів 2 орхдеї різних кольорів та 5 кущів хризантем п'яти різних кольорів. Усього в магазині є в продажу орхідеї 10 кольорів та кущі хризантем 8 кольорів. Скільки всього є способів формування такого замовлення?

\noindent\rule[0.5ex]{\linewidth}{1pt}
Кількість способів обрати 2 орхідеї з 10 можна обчислити за формулою числа комбінацій (порядок не має значення).
$C_n^m=\dfrac{n!}{m!\cdot(n-m)!}$.
$C_{10}^2=\dfrac{10!}{2!\cdot(10-2)!}=\dfrac{10\cdot9}{2\cdot1}=45$.
Аналогічно обсилюється кількість варіантів вибору 5 кущів хризантем з 8 можливих.
$C_{8}^5=\dfrac{8!}{5!\cdot(8-3)!}=\dfrac{8\cdot7\cdot6}{3\cdot2\cdot1}=56$.
Оскільки обрати потрібно одночасно і обидва типи квітів, то застосовуємо формулу множення $45\cdot56=\textbf{2520}$.


\end{document}
